\chapter{Leading Design of the Payload}

	\section{Drone Structures}\label{PL:Design:Structures}
		\subsection{Arm Configuration}
			This is a filler paragraph

		\subsection{Landing Legs}
			This is a filler paragraph
	
		\subsection{Ice Recovery and Mobility Agent}
		The pin joint design was chosen because the design has the best sturdiness, manufacturability, and because of its absence of a sliding motion. The bolt in the pin joint mechanism makes the retrieval mechanism sturdy in the direction going into and out of the page, perpendicular to the scooping motion,  when looking at Figure INSERT FIGURE OF CAD OF PIN JOINT. The rod connecting the drone to the scoop mechanism keeps a point on the scoop fixed to the drone at all times so the scoop is not flimsy. The connecting rod creates sturdiness in the direction parallel to the scooping motion because the rod is a fixed length. This allows control over how much the scoop can open or close based on how high the bottom of the connecting rod is on the bolt. Every part on the pin joint mechanism is thick with few potential points for failure. 

		The pin joint mechanism was also chosen due to its ability to convert rotational motion into translational motion. Instead of moving a mechanism up and down, which would not be feasible given the space inside the rocket, the team decided that a rotating motion to actuate the scoop would be ideal. This way there can just be a servo at the top instead of creating a new mechanism to slide up and down a railing. The team decided that having a nut on the bolt would be easiest to manufacture and simplest because it requires the least amount of moving parts, and the tolerances are easiest to fit into the bolt. The pin joint idea also allows for the easiest movement without friction because the moving part is the bolt which allows the mechanism to stay but move easily on command.

		The biggest risk to the current design is that the pin joint mechanism relies on an electrical servo which creates another method of failure if the avionics fail to perform as intended. Another problem is that the pin doesn’t entirely eliminate friction. The pin joint mechanism could theoretically get stuck if the servo force is not as strong as the frictional force on the pin. Another disadvantage is that there is little material on the other side of the pin from the scoop connecting the rod to the scoop. To compensate for this, strong material will be used like metal, and there will be testing performed to ensure that there is enough material with a safety factor.



	\section{Avionics}\label{PL:Design: Avionics}
		\subsection{Flight Controller}
			This is a filler paragraph

		\subsection{Computer Vision}
			This is a filler paragraph
