\chapter{Executive Summary}

Over the period spanning from proposal submission to \gls{pdr} submission (August--November 2019), the team has made considerable progress in shaping a preliminary vehicle and payload design. Throughout these developments, the project chiefly guided by the the following main goals: safety, feasibility and timeliness. This part will outline the present mission design in relation to these goals, as well as the overall considerations that shaped the leading launch vehicle and payload design.

The launch vehicle is slated to be a single-booster rocket, weighing \SI{10.68}{\kilo\gram} (\SI{23.55}{\poundm}), with a base-to-tip length of \SI{2.41}{\meter} (\SI{7}{\feet} \SI{11}{\inch}). The vehicle caliber is two-fold, as a transition will be utilized so as to attain a wider fairing diameter; the body diameter is \SI{10.16}{\centi\meter} (\SI{4}{\inch}), whereas the fairing diameter is \SI{15.24}{\centi\meter} (\SI{6}{\inch}). In designing the vehicle, a reference altitude of \SI{1.5}{\kilo\meter} (\SI{4921}{\feet}) was aimed at, prompting the appointment of the AeroTech K780R as the motor of choice. This motor has a burn time of \SI{3}{\second}, and produces a total impulse of \SI{2371}{\newton\second} (L-class). The nose cone was chosen to be of tangent ogive shape, with an aspect ratio of 3:1.