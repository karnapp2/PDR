\chapter{Executive Summary}

Over the period spanning from proposal submission to \gls{pdr} submission (August--November 2019), the team has made considerable progress in shaping a preliminary vehicle and payload design. Throughout these developments, the project chiefly guided by the the following main goals: safety, feasibility and timeliness. This part will outline the present mission design in relation to these goals, as well as the overall considerations that shaped the leading launch vehicle and payload design.

The launch vehicle is slated to be a single-booster rocket, weighing \SI{10.68}{\kilo\gram} (\SI{23.55}{\poundm}), with a base-to-tip length of \SI{2.41}{\meter} (\SI{7}{\feet} \SI{11}{\inch}). The vehicle caliber is two-fold, as a transition will be utilized so as to attain a wider fairing diameter; the body diameter is \SI{10.16}{\centi\meter} (\SI{4}{\inch}), whereas the fairing, or payload bay, diameter is \SI{15.24}{\centi\meter} (\SI{6}{\inch}). In designing the vehicle, a reference altitude of \SI{1.5}{\kilo\meter} (\SI{4921}{\feet}) was aimed at, prompting the appointment of the AeroTech K780R as the motor of choice. This motor has a burn time of \SI{3}{\second}, and produces a total impulse of \SI{2371}{\newton\second} (L-class). The nose cone was chosen to be of tangent ogive shape, with an aspect ratio of 3:1.

In reference to recovery of the vehicle, which forms a sizable portion of the mission's consideration in terms of both safety and system reusability, two parachutes are utilized; a \SI{50.8}{\centi\meter} (\SI{20}{\inch}) diameter drogue parachute deployed at \SI{800}{\feet}, and a \SI{183}{\centi\meter} (\SI{6}{\feet}) diameter main parachute deployed at one second after apogee. Thus, the system will have a total of four segments: the booster (or lower) stage, the switchband, the upper stage, and the nose cone. Seeing as the \gls{uav} will not remain tethered nor lowered by parachute, it is not accounted for as a separate section. The final descent velocity of the vehicle is approximately \SI{7}{\meter\per\second}, thereby satisfying the maximum terminal kinetic energy requirement.

In addition to the launch vehicle, a payload is necessary to effectuate the mission requirements, which state that a ground-based granular ice sample is to be retrieved and displaced by \SI{10}{\feet} (\SI{3.05}{\meter}). The present design is based on a \gls{uav} with four arms, each having one engine and rotor. The aforestated payload bay allows for the payload to sit in the up right position with respect to the vehicle's descent attitude, such that upon deployment the system may immediately function as intended. To allow the \gls{uav} to descent unobstructed, the nose cone will be jettisoned by means of a high impulse carbon dioxide gas release, so as to force the nose cone parachute out. Upon jettison, this parachute will deploy and guide the nose cone down to a sufficiently low descent velocity. In transit, the \gls{uav} was secured in its stowed position using a solenoid-based mechanism. Lowering the payload requires a winch to unfurl after the solenoids have been disengaged, allow for the release of the payload which will slide out of the payload bay on four rails. Following these events, the \gls{uav} will be lowered using a winch system, after having released the  This deployment will be commanded by way of a the rocket avionics system, which will command nose cone jettison